\documentclass{scrartcl}

\usepackage{mdwlist}
\usepackage[scaled]{helvet}
\usepackage{fullpage}
\usepackage{polski}
\usepackage[utf8x]{inputenc}
\usepackage{color}
\usepackage{mathtools}
\usepackage{fancyhdr}
\usepackage{graphicx}
\usepackage[unicode=true]{hyperref}
\usepackage{multirow}
\usepackage[table]{xcolor}
\usepackage{listings}

\begin{document}

\sloppy

\newcommand{\ttt}{$T^3\:$}

\title{\ttt - Twórca Tablic Tęczowych}
\subtitle{Równoległe wyznaczanie tęczowych tablic (``rainbow tables'') w~zgadnieniach kryptografii dla haseł zaszyfrowanych algorytmem DES: Scala}
\author{Bartosz Pieńkowski, Barnaba Turek}
\maketitle
\section{Opis}
\ttt to zestaw programów wyznaczających tablice tęczowe i~pozwalających sprawdzić poprawność ich wyznaczenia (przez wyznaczenie funkcji skrótu dla danego ciągu znaków (dalej klucza) i~próbę odwrócenia tego procesu).

\emph{Tablice tęczowe} to sposób przechowywania wcześniej obliczonych danych pozwalających odwracać (analitycznie nieodwracalną) funkcję skrótu\footnote{inaczej kryptograficzną funkcję mieszającą, ang. \emph{hash function}}.
\emph{Tablice tęczowe} pozwalają zmniejszyć wymagania dyskowe (w~stosunku do prostego zapisywania wszystkich par klucz\dywiz f(klucz)) kosztem wymagań obliczeniowych.
Osiągane to jest za pomocą tworzenia tzw. łańcuchów skrótów\footnote{ang. \emph{hash chaining}} i~zapisywaniu tylko pierwszego i~ostatniego elementu łańcucha.

\subsection{Funkcja skrótu}
\ttt wyznacza \emph{tablice tęczowe} dla funkcji skrótu zgodnej z~funkcją crypt (należącą do standardu \textbf{POSIX}) działającej w~oparciu o~standard \textbf{DES}.

\section{Użycie}
\subsection{Tworzenie tablic tęczowych}
Program generujący tablice tęczowe nazywa się \emph{t3}.

\subsubsection{Wywołanie programu}
Użytkownik programu t3 podaje trzy argumenty linii poleceń. Pierwszy argument określa długość klucza, dla którego mają być wygenerowane \emph{tablice tęczowe} (od 1 do 8).
Drugi argument określa długość obliczanych łańcuchów.
Trzeci argument jest opcjonalny i~określa alfabet użyty do generowania kluczy. Domyślnie alfabet to małe litery.

\subsubsection{Wyjście programu}
Tablice tęczowe zostaną zapisane w~aktualnym katalogu.
Program tworzy plik, składający się z~wierszy. Każdy wiersz składa się z~początkowego i~końcowego elementu łańcucha skrótów, oddzielnoych spacją.

\subsubsection{Przykładowe wywołania}
Wywołanie:
\begin{lstlisting}
  $ t3 2 10 abc
\end{lstlisting}
Wygeneruje tablice tęczowe dla haseł o~dłguości dwóch znaków z~alfabetu abc. Łańcuchy skrótów będą miały długość 10.

Aby uprościć podawanie alfabetu można wykorzystać inne, dostępne w~systemie narzędzia. Np. w~wielu powłokach systemu GNU/Linux można wykorzystać program \textbf{perl} w~następujący sposób:
\begin{lstlisting}
  $ t3 2 10 `perl -e 'print join ``, a..z,A..Z,0..9'`
\end{lstlisting}
Wywołanie takie jest równoważne wywołaniu:
\begin{lstlisting}
  $ t3 2 10 abcdefghijklmnopqrstuvwxyzABCDEFGHIJKLMNOPQRSTUVWXYZ0123456789
\end{lstlisting}
i~generuje tablice tęczowe dla dwuznakowych haseł składających się z~kombinacji wielkich liter, małych liter i~cyfr.
\subsection{Generowanie skrótu}
Do generowania i~skrótów służy program \emph{t3-hash}.
\subsubsection{Wywołanie programu}
Użytkownik programu \emph{t3-hash} jako argument podaje klucz (1 do 8 znaków), dla którego ma być wygenerowany skrót.
Następnie program wypisuje skrót na standardowe wyjście.


\subsection{Odwracanie funkcji skrótu}
Do odwracania funkcji skrótu służy program \emph{t3-reverse}.
\subsubsection{Wywołanie programu}
Użytkownik programu \emph{t3-reverse} jako argument podaje wartość funkcji skrótu, dla której znaleziona ma być wartość klucza.

Jeżeli program został wywołany z~katalogu, w~którym nie ma wygenerowanych tablic tęczowych, program zakończy działanie wypisując informację o~braku tablic.

Jeżeli program nie znajdzie klucza pasującego do zadanej wartości funkcji skrótu, program zakończy działanie wypisując informację o~niepowodzeniu.

Jeżeli działanie programu zakończy się sukcesem, program wypisze znaleziony klucz. Poprawność znalezionego klucza będzie można sprawdzić korzystając z~programu \emph{t3-hash}.

\section{Rozwiązania}
Program zostanie wykonany w~języku \textbf{Scala} na platformę \textbf{JVM}.

Programy \emph{t3-hash} i~\emph{t3-reverse} nie będą działać współbieżnie - programy te służą głównie do sprawdzania poprawności wygenerowanych tablic i~są znacząco mniej wymagające obliczeniowo.

Zrównoleglenie wyznaczania tablic tęczowych zostanie osiągnięte za pomocą mechanizmu Aktorów oferowanego przez język \textbf{Scala}.
Mechanizm ten jest zrealizowany na wirtualnej maszynie Javy za pomocą wątków.

Zamierzamy wykorzystać komunikację globalną - jeden wątek będzie zarządzał wszystkimi innymi wątkami.

Naszym zdaniem najlepszą dekompozycją problemu przy tak postawionym zadaniu będzie dekompozycja domenowa, tj. równomierny podział początkowych\footnote{zaczynających łańcuch skrótów} kluczy na wątki.

\end{document}
